\documentclass[11pt,a4paper]{article}

\usepackage[utf8]{inputenc}
\usepackage{fullpage}
\usepackage{authblk}
\usepackage{graphicx}
\usepackage{caption}
\usepackage{subcaption}
\usepackage{amsmath}
\usepackage{amssymb}
\usepackage{bbm}
\usepackage{euscript}
\usepackage{graphicx}
\usepackage{subcaption}
\usepackage{lmodern}
\usepackage{textcomp}
\usepackage{titlesec}
\usepackage{float}
\usepackage{amsfonts}
\usepackage{bm}
\usepackage{stmaryrd}
\usepackage{bbm}

% red
% R: 202 G: 53 B: 66 
% \definecolor{myred}{rgb}{0.789,0.207,0.2578}

% blue
% R: 39 G: 100 B: 123
% \definecolor{myhyperblue}{rgb}{0.152,0.39,0.48}

% green
% R: 132 G: 159 B: 173
% \definecolor{mygreen}{rgb}{0.5156,0.62,0.6757}

\usepackage{color}
\definecolor{myorange}{rgb}{0.9568,0.4941,0.1961}
\definecolor{myred}{rgb}{0.9098,0.1294,0.2078}
\definecolor{myblue}{rgb}{0.0352,0.4981,0.6509}
\definecolor{mygreen}{rgb}{0.2235,0.6353,0.2588}
\definecolor{lightgray}{rgb}{0.8,0.8,0.8}
\definecolor{myhyperblue}{rgb}{0.1607,0.3922,0.9}
\definecolor{mygrey}{rgb}{0.3,0.3,0.3}

\newcommand{\unknown}[1]{\bm{{\color{myred}{#1}}}}
\newcommand{\param}[1]{\bm{{\color{myhyperblue}{#1}}}}
\newcommand{\data}[1]{\bm{{\color{mygreen}{#1}}}}
\newcommand{\keyword}[1]{[\texttt{\textbf{#1}}]\!\,}

\title{Functional scope \texttt{wave1D v0}}

\author[1]{Florian Le Boudrais, Alexandre Imperiale}

\begin{document}

% \bibliographystyle{apalike}

\maketitle

\section{Continuous models}
\subsection{Strong formulations}
Let $\Omega = ]0; L[$ be the one-dimensional domain of interest and $\partial\Omega = \{0, L\}$ its boundary, we consider the one-dimensional model
\begin{equation*}
\quad \quad
\keyword{Elastic}~
\left\lbrace
\begin{aligned}
& \param{\alpha}\partial^2_{tt} \unknown{u} - \partial_x\big( \param{\beta} \partial_x \unknown{u} \big) = \data{f},\quad \text{in }\Omega,\\
& \mathcal{B}_0(\unknown{u}, \partial_t \unknown{u}, \partial_x \unknown{u}) = \data{g_0},\quad \text{at } x = 0,\\
&\mathcal{B}_L(\unknown{u}, \partial_t \unknown{u}, \partial_x \unknown{u}) = \data{g_L},\quad \text{at } x = L,\\
& \unknown{u}(\cdot, 0) = \data{u_0},\quad \partial_t\unknown{u}(\cdot, 0) = \data{u_1},
\end{aligned}
\right.
\end{equation*}
where $\data{u_0}(x)$ and $\data{u_1}(x)$ are input initial conditions, $\data{f}(x, t)$ and $\data{g_{0,L}}(t)$ are three input source terms, $\param{\alpha}(x) > 0$ and $\param{\beta}(x) > 0$ are two input model parameters and $\mathcal{B}_{0, L}(\cdot, \cdot, \cdot)$ are two ``boundary condition'' operators taking the following forms
\begin{itemize}
\item[] \keyword{Dirichlet} boundary condition: 
\begin{equation*}
\forall s \in\{0, L\},\quad \mathcal{B}_s(\unknown{u}, \partial_t \unknown{u}, \partial_x \unknown{u}) = \unknown{u}(s, t),
\end{equation*}
\item[] \keyword{Robin} boundary condition: 
\begin{equation*}
\forall s \in\{0, L\},\quad \mathcal{B}_s(\unknown{u}, \partial_t \unknown{u}, \partial_x \unknown{u}) = \param{\gamma}\unknown{u}(s, t) + \partial_x \unknown{u}(s, t) n(s),
\end{equation*}
\item[] \keyword{Absorbing} boundary condition:
\begin{equation*}
\forall s \in\{0, L\},\quad \mathcal{B}_s(\unknown{u}, \partial_t \unknown{u}, \partial_x \unknown{u}) = \param{\gamma} \partial_t \unknown{u}(s, t) + \partial_x \unknown{u}(s, t) n(s),
\end{equation*}
\end{itemize}
with $\param{\gamma}(x) \geq 0$ an input model parameter, and $n(0) = -1$ and $n(L) = 1$.
\subsection{Weak formulations}
We denote by $V = H^1(\Omega)$ and $V_0 = H^1_0(\Omega)$ the standard Sobolev spaces, and the spaces associated to potential non-homogeneous Dirichlet boundary conditions
\begin{equation*}
V_{\data{g_0}} = \big\{v \in V,~v(0) = \data{g_0}\big\},\quad V_{\data{g_L}} = \big\{v \in V,~v(L) = \data{g_L}\big\}
\end{equation*}
Denoting by
\begin{equation*}
m_{\param{\alpha}}(u, v) = \int_\Omega \param{\alpha} u v~\mathrm{d}\Omega, \quad k_{\param{\beta}}(u, v) = \int_\Omega \param{\beta} \partial_xu \,\partial_x v~\mathrm{d}\Omega,
\end{equation*}
we consider the following weak formulations
\begin{itemize}
\item[] \keyword{Elastic - Dirichlet} For any $t > 0$, find $\unknown{u}\in V_{\data{g_0}} \cap V_{\data{g_L}}$ such that for any $u^* \in V_{\data{g_0}} \cap V_{\data{g_L}}$
\begin{equation*}
\dfrac{\mathrm{d}^2}{\mathrm{d}t^2} m_{\param{\alpha}}(\unknown{u}, u^*) +
k_{\param{\beta}} (\unknown{u}, u^*) = \int_\Omega \data{f}u^*~\mathrm{d}\Omega, 
\end{equation*}
\item[] \keyword{Elastic - Robin} For any $t > 0$, find $\unknown{u}\in V$ such that for any $u^* \in V$
\begin{equation*}
\dfrac{\mathrm{d}^2}{\mathrm{d}t^2} m_{\param{\alpha}}(\unknown{u}, u^*) +
k_{\param{\beta}} (\unknown{u}, u^*)  + \sum_{s\in\{0, L\}}\param{\gamma}(s)\unknown{u}(s, t)u^*(s) = \int_\Omega \data{f}u^*~\mathrm{d}\Omega + \sum_{s\in\{0, L\}}\data{g_s}(t)u^*(s), 
\end{equation*}
\item[] \keyword{Elastic - Absorbing} For any $t > 0$, find $\unknown{u}\in V$ such that for any $u^* \in V$
\begin{equation*}
\dfrac{\mathrm{d}^2}{\mathrm{d}t^2} m_{\param{\alpha}}(\unknown{u}, u^*) +
k_{\param{\beta}} (\unknown{u}, u^*)  + \sum_{s\in\{0, L\}}\param{\gamma}(s)\partial_t\unknown{u}(s, t)u^*(s) = \int_\Omega \data{f}u^*~\mathrm{d}\Omega + \sum_{s\in\{0, L\}}\data{g_s}(t)u^*(s), 
\end{equation*}
\end{itemize}
completed with the initial conditions $\unknown{u}(\cdot, 0) = \data{u_0}$ and $\partial_t\unknown{u}(\cdot, 0) = \data{u_1}$. 

\subsection{Energy conservation}

We denote, for any $\unknown{u}\in V$ and any time $t > 0$, the energy functional
\begin{equation*}
\mathcal{E}_{\param{\alpha},\param{\beta}}(\unknown{u}) = \dfrac{1}{2}\Big\{m_{\param{\alpha}}(\partial_t \unknown{u}, \partial_t \unknown{u}) +  k_{\param{\beta}}(\unknown{u}, \unknown{u})\Big\}.
\end{equation*}
One can verify the following conservation properties in the different weak problems stated previously
\begin{itemize}
\item[] \keyword{Elastic - Dirichlet} Denoting by $\mathcal{R}(\cdot)$ any given harmonic lifting operator such that $\mathcal{R}(\data{g_{0, L}}) \in V$ are the lifted Dirichlet data, and $\unknown{\widetilde{u}} = \unknown{u} - \sum_{s\in\{0, L\}}\mathcal{R}(\data{g_{s}}) \in V_0$ then we have
\begin{equation*}
\dfrac{\mathrm{d}}{\mathrm{d}t} \mathcal{E}_{\param{\alpha},\param{\beta}}(\unknown{\widetilde{u}}) = \int_\Omega \data{f}\partial_t \unknown{\widetilde{u}}~\mathrm{d}\Omega -\sum_{s\in\{0, L\}}  \Big\{m_{\param{\alpha}}(\partial^2_{tt} \mathcal{R}(\data{g_{s}}), \partial_t \unknown{\widetilde{u}}) + k_{\param{\beta}}(\mathcal{R}(\data{g_{s}}), \partial_t \unknown{\widetilde{u}})\Big\},
\end{equation*}
\item[] \keyword{Elastic - Robin}
\begin{equation*}
\dfrac{\mathrm{d}}{\mathrm{d}t}\Big\{\mathcal{E}_{\param{\alpha},\param{\beta}}(\unknown{u}) +  \dfrac{1}{2}\sum_{s\in\{0, L\}}\param{\gamma}(s)|\unknown{u}(s, t)|^2 \Big\} = \int_\Omega \data{f}\partial_t \unknown{u}~\mathrm{d}\Omega + \sum_{s\in\{0, L\}}\data{g_s}(t)\partial_t\unknown{u}(s, t),
\end{equation*}
\item[] \keyword{Elastic - Absorbing}
\begin{equation*}
\dfrac{\mathrm{d}}{\mathrm{d}t} \mathcal{E}_{\param{\alpha},\param{\beta}}(\unknown{u}) = \int_\Omega \data{f}\partial_t \unknown{u}~\mathrm{d}\Omega + \sum_{s\in\{0, L\}}\data{g_s}(t)\partial_t\unknown{u}(s, t) - \sum_{s\in\{0, L\}}\param{\gamma}(s)|\partial_t\unknown{u}(s, t)|^2.
\end{equation*}
\end{itemize}

\section{Space discretization}
\subsection{Construction of high-order finte element spaces}
We decompose $\Omega$ in a set of $N_e$ edges denoted by $\mathcal{E}^\delta$ and such that
\begin{equation*}
\mathcal{E} = \{e_l\}_{l=1}^{N^\delta_e}, \quad\quad \forall l\neq k\quad \overset{\circ}{e}_l \cap \overset{\circ}{e}_k = \varnothing, \quad\quad \Omega = \bigcup_{l=1}^{N^e}e_l.
\end{equation*}
We define $V_h$ the finite element space by
\begin{equation*}
V_h = \big\{ v_h \in \mathcal{C}^0(\overline{\Omega}), \quad\forall e \in \mathcal{E} \quad v_h|_{e} \in \mathcal{P}^k(e) \big\} \subset V,
\end{equation*}
where $k$ is the order of approximation, assumed to be identical for every edges in $\mathcal{E}$. Denoting by $\widehat{e} = [0; 1]$ the reference edge, and the affine transforms $F_l$ such that
\begin{equation*}
e_l = F_l(\widehat{e}), \quad \forall l=1,\cdots, N_e,
\end{equation*}
one have the equivalent definition of the approximation space
\begin{equation*}
V_h = \big\{ v_h \in \mathcal{C}^0(\overline{\Omega}), \quad\forall l=1,\cdots,N_e\quad  \exists! \widehat{v}_{h, l} \in \mathcal{P}^k(\widehat{e}), \quad v_h|_{e_l} = \widehat{v}_{h, l} \circ  F_l^{-1} \big\}.
\end{equation*}
On the reference edges, we set $\widehat{n}=k+1$ nodes denoted by
\begin{equation*}
\widehat{\Xi} = \{\widehat{\xi}_i\}_{i=1}^{\widehat{n}} \subset \widehat{e},
\end{equation*}
and satisfying the following extremity constraints
\begin{equation*}
\widehat{\xi}_1 = 0, \quad \widehat{\xi}_{\widehat{n}} = 1.
\end{equation*}
We associated to these nodes the Lagrange polynomials $\{\widehat{\varphi}_i\}_{i=1}^{\widehat{n}}$ forming a basis of  $\mathcal{P}^k(\widehat{e})$ and such that
\begin{equation*}
\widehat{\varphi}_i(\widehat{\xi}_j) = \delta_{ij}, \quad \forall i,j=1, \cdots, \widehat{n}.
\end{equation*}
The set of nods obtained from applying the transformations  $\{F_l\}_{l=1}^{N_e}$ to the nodes $\widehat{\Xi}$ is denoted by
\begin{equation*}
\Xi = \{\xi_I\}_{I=1}^{N}\subset\overline{\Omega},
\end{equation*}
from which we have discarded redundant nodes at the edges' extremities. The global number of nodes $N$ is given by
\begin{equation*}
N = \mathrm{card}(\Xi) = N_e \times k + 1.
\end{equation*}
To the set of nodes $\Xi$ we can associated the set of Lagrange basis functions $\{\varphi_I\}_{I=1}^{N}$ satisfying
\begin{equation*}
\varphi_I(\xi_J) = \delta_{IJ}, \quad \forall I,J=1,\cdots,N.
\end{equation*}
These Lagrange functions are locally represented by the local Lagrange polynomials i.e.
\begin{equation*}
\forall I = 1,\cdots, N, \quad \forall l \in \llbracket 1; N_e \rrbracket,\text{ s.t. } \xi_I \in e_l ,\quad \exists! i \in \llbracket 1; \widehat{n} \rrbracket,\quad \varphi_I|_{e_l} = \widehat{\varphi}_i \circ F_l^{-1},
\end{equation*}
where we have implicitly defined the ``local-to-global'' index transformation, denoted by $\ell_g(\cdot, \cdot)$ and defined as
\begin{equation*}
\ell_g(\cdot, \cdot)~:~\llbracket 1;\widehat{n} \rrbracket\times \llbracket 1;N_e \rrbracket \mapsto \llbracket 1;N \rrbracket, \quad I = \ell_g(i, l).
\end{equation*}
In the functional scope of \texttt{wave1D v0}, we consider two types of nodes on the reference edges \keyword{Equally Distributed} and  \keyword{Gauss-Lobatto}.

\subsection{Finite element operators}
\subsubsection{Mass operator}
We define the discrete mass operator $\mathbb{M}_{\param{\alpha}} \in \mathcal{M}_{N\times N}(\mathbb{R})$ as
\begin{equation*}
\forall I,J = 1,\cdots, N,\quad \mathbb{M}_{\param{\alpha}, IJ} = m_{\param{\alpha}}(\varphi_I, \varphi_J) = \int_\Omega \param{\alpha} \varphi_I\varphi_J~\mathrm{d}\Omega,
\end{equation*}
and is available in two different forms
\begin{itemize}
\item[] \keyword{Assembled}: $\mathbb{M}_{\param{\alpha}}$ is stored into a \keyword{SparseMatrix} data structure, and computed using an assembling procedure with a sufficiently precise quadrature formula for the computation of the local mass matrices,
\item[] \keyword{Lumped}:  $\mathbb{M}_{\param{\alpha}}$ is stored into a \keyword{Vector} data structure, and computed using an assembling procedure with an approximation of the local mass matrices referred to as the ``mass lumping'' technique.
\end{itemize}
We consider exclusively the \keyword{Gauss-Lobatto} quadrature formula for computing the integrals in the local mass matrices. For both modes, the available operations are \keyword{MltAdd} and \keyword{Inv}.

\subsubsection{Stiffness operator}
We define the discrete stiffness operator $\mathbb{K}_{\param{\beta}} \in \mathcal{M}_{N\times N}(\mathbb{R})$ as
\begin{equation*}
\forall I,J = 1,\cdots, N,\quad \mathbb{K}_{\param{\beta}, IJ} = k_{\param{\beta}}(\varphi_I, \varphi_J) = \int_\Omega \param{\beta}\varphi_I^\prime\,\varphi_J^\prime~\mathrm{d}\Omega,
\end{equation*}
and is available in two different forms
\begin{itemize}
\item[] \keyword{Assembled}: $\mathbb{K}_{\param{\alpha}}$ is stored into a \keyword{SparseMatrix} data structure, and computed using an assembling procedure with a sufficiently precise quadrature formula for the computation of the local stiffness matrices,
\item[] \keyword{Locally Assembled}: Every local stiffness matrices are assembled into $N_e$ \keyword{FullMatrix} data structures. Compared to  the \keyword{Assembled} mode, this enable application of $\mathbb{K}_{\param{\alpha}}$ in parallel, but increases the memory footprint.
\end{itemize}
The available operation for the stiffness operator is \keyword{MltAdd} exclusively since, by construction, the stiffness operator has a non-empty kernel thus is not invertible.

\section{\keyword{ExplicitOrderTwo} time discretization}
Depending on the boundary conditions we consider the following discrete schemes
\subsection{The case of \keyword{Dirichlet} boundary conditions}
We use a pseudo-elimination strategy to incorporate the boundary conditions in the standard leap-frog scheme. The scheme is decomposed into two steps
\begin{itemize}
\item[(i)] Computing right-hand side
\begin{equation*}
\overrightarrow{\unknown{U}}^* \longleftarrow \Delta t^2 \mathbb{M} \overrightarrow{\data{F}}^n + \Big\{ 2 \mathbb{M}_{\param{\alpha}} - \Delta t^2 \mathbb{K}_{\param{\beta}} \Big\} \overrightarrow{\unknown{U}}^{n} - \mathbb{M}_{\param{\alpha}} \overrightarrow{\unknown{U}}^{n-1},
\end{equation*}
\item[(ii)] Solving linear system
\begin{equation*}
\overrightarrow{\unknown{U}}^* \longleftarrow \widetilde{\mathbb{M}}_{\param{\alpha}}^{-1} \widetilde{\overrightarrow{\unknown{U}}^*},
\end{equation*}
\end{itemize}
where $\widetilde{\mathbb{M}}_{\param{\alpha}}$ and $\widetilde{\overrightarrow{\unknown{U}}}^*$ are resulting from a pseudo-elimination procedure.

\subsection{The case of \keyword{Robin} boundary conditions}
The conservative time scheme used to incorporate the boundary conditions is the following
\begin{equation*}
\mathbb{M}_{\param{\alpha}}\dfrac{\overrightarrow{\unknown{U}}^{n+1} - 2\overrightarrow{\unknown{U}}^{n} + \overrightarrow{\unknown{U}}^{n-1}}{\Delta t^2} + \sum_{s\in\{0, L\}} \param{\gamma}(s)\mathbbm{1}_s \dfrac{\overrightarrow{\unknown{U}}^{n+1} + \overrightarrow{\unknown{U}}^{n-1}}{2} + \mathbb{K}_{\param{\beta}} \overrightarrow{\unknown{U}}^{n} = \mathbb{M} \overrightarrow{\data{F}}^n + \sum_{s\in\{0, L\}}  \overrightarrow{\data{G_s}}^n,
\end{equation*}
where $\mathbbm{1}_{0, L} \in \mathcal{M}_{N\times N}(\mathbb{R})$ are matrices defined as $(\mathbbm{1}_{0})_{ij} = \delta_{i=1, j=1}$ and $(\mathbbm{1}_{L})_{ij} = \delta_{i=N, j=N}$. This scheme is decomposed into two steps
\begin{itemize}
\item[(i)] Computing right-hand side
\begin{equation*}
\overrightarrow{\unknown{U}}^* \longleftarrow \Delta t^2 \mathbb{M} \overrightarrow{\data{F}}^n + \Delta t^2 \sum_{s\in\{0, L\}}  \overrightarrow{\data{G_s}}^n + \Big\{ 2 \mathbb{M}_{\param{\alpha}} - \Delta t^2 \mathbb{K}_{\param{\beta}} \Big\} \overrightarrow{\unknown{U}}^{n} - \Big\{ \mathbb{M}_{\param{\alpha}} + \dfrac{\Delta t^2}{2} \sum_{s\in\{0, L\}} \param{\gamma}(s)\mathbbm{1}_s \Big\} \overrightarrow{\unknown{U}}^{n-1},
\end{equation*}
\item[(ii)] Solving linear system
\begin{equation*}
\overrightarrow{\unknown{U}}^* \longleftarrow \Big\{\mathbb{M}_{\param{\alpha}} + \dfrac{\Delta t^2}{2} \sum_{s\in\{0, L\}} \param{\gamma}(s)\mathbbm{1}_s \Big\}^{-1} \overrightarrow{\unknown{U}}^*,
\end{equation*}
\end{itemize}

\subsection{The case of \keyword{Absorbing} boundary conditions}
The dissipative time scheme used to incorporate the boundary conditions is the following
\begin{equation*}
\mathbb{M}_{\param{\alpha}}\dfrac{\overrightarrow{\unknown{U}}^{n+1} - 2\overrightarrow{\unknown{U}}^{n} + \overrightarrow{\unknown{U}}^{n-1}}{\Delta t^2} + \sum_{s\in\{0, L\}} \param{\gamma}(s)\mathbbm{1}_s \dfrac{\overrightarrow{\unknown{U}}^{n+1} - \overrightarrow{\unknown{U}}^{n-1}}{2\Delta t} + \mathbb{K}_{\param{\beta}} \overrightarrow{\unknown{U}}^{n} = \mathbb{M} \overrightarrow{\data{F}}^n + \sum_{s\in\{0, L\}}  \overrightarrow{\data{G_s}}^n,
\end{equation*}
This scheme is decomposed into two steps
\begin{itemize}
\item[(i)] Computing right-hand side
\begin{equation*}
\overrightarrow{\unknown{U}}^* \longleftarrow \Delta t^2 \mathbb{M} \overrightarrow{\data{F}}^n + \Delta t^2 \sum_{s\in\{0, L\}}  \overrightarrow{\data{G_s}}^n + \Big\{ 2 \mathbb{M}_{\param{\alpha}} - \Delta t^2 \mathbb{K}_{\param{\beta}} \Big\} \overrightarrow{\unknown{U}}^{n} - \Big\{ \mathbb{M}_{\param{\alpha}} - \dfrac{\Delta t}{2} \sum_{s\in\{0, L\}} \param{\gamma}(s)\mathbbm{1}_s\Big\} \overrightarrow{\unknown{U}}^{n-1},
\end{equation*}
\item[(ii)] Solving linear system
\begin{equation*}
\overrightarrow{\unknown{U}}^* \longleftarrow \Big\{\mathbb{M}_{\param{\alpha}} + \dfrac{\Delta t}{2} \sum_{s\in\{0, L\}} \param{\gamma}(s)\mathbbm{1}_s \Big\}^{-1} \overrightarrow{\unknown{U}}^*,
\end{equation*}
\end{itemize}

\subsection{Order one initial conditions}
In the case of \keyword{OrderOneIC}, the discrete propagators are simply initialized using
\begin{equation*}
\overrightarrow{\unknown{U}}^0 = \overrightarrow{\data{U}_0}, \quad \overrightarrow{\unknown{U}}^1 = \Delta t \overrightarrow{\data{U}_1}+ \overrightarrow{\data{U}_0}.
\end{equation*}

\subsection{Order two initial conditions}
In the case of \keyword{OrderTwoIC}, we consider the following relations
\begin{equation*}
\overrightarrow{\unknown{U}}^0 = \overrightarrow{\data{U}_0}, \quad \dfrac{\overrightarrow{\unknown{U}}^1 - \overrightarrow{\unknown{U}}^{-1}}{2\Delta t} = \overrightarrow{\data{U}_1},
\end{equation*}
completed with the following discrete propagator
\begin{equation*}
\mathbb{M}_{\param{\alpha}}\dfrac{\overrightarrow{\unknown{U}}^{1} - 2\overrightarrow{\unknown{U}}^{0} + \overrightarrow{\unknown{U}}^{-1}}{\Delta t^2} + \mathbb{K}_{\param{\beta}} \overrightarrow{\unknown{U}}^{0} = 0,
\end{equation*}
where we assume that the initial conditions have a compact support in the open domain $\Omega$, hence we discard the boundary conditions terms. Combining these relations we obtain
\begin{equation*}
\overrightarrow{\unknown{U}}^0 = \overrightarrow{\data{U}_0}, \quad \overrightarrow{\unknown{U}}^1 = \Delta t \overrightarrow{\data{U}_1} + \Big\{\mathbb{I} - \dfrac{\Delta t^2}{2} \mathbb{M}_{\param{\alpha}}^{-1}\mathbb{K}_{\param{\beta}}\Big\}\overrightarrow{\data{U}_0}.
\end{equation*}


\end{document}
